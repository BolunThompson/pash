\documentclass[sigplan, screen, review, anonymous]{acmart}
\acmSubmissionID{XXX}
\renewcommand\footnotetextcopyrightpermission[1]{}
\settopmatter{printfolios=false,printccs=false,printacmref=false}

% \usepackage{amssymb}
% \usepackage{cleveref}
\usepackage{enumitem}
\usepackage{booktabs}
\usepackage{soul}
\usepackage{xspace}
\usepackage{color}
\usepackage{xcolor}
\usepackage{upquote}
\usepackage{listings}
\usepackage{amsmath}
\usepackage{wrapfig}
\usepackage{syntax}
\usepackage{caption}

\captionsetup[figure]{font=footnotesize,name={Fig.},labelfont={bf, footnotesize}}
\captionsetup[table]{font=footnotesize,name={Tab.},labelfont={bf, footnotesize}, skip=2pt, aboveskip=2pt}
\captionsetup{font=footnotesize,labelfont={bf, footnotesize}, belowskip=2pt}

\newcommand{\eg}{{\em e.g.}, }
\newcommand{\ie}{{\em i.e.}, }
\newcommand{\etc}{{\em etc.}\xspace}
\newcommand{\vs}{{\em vs.} }
\newcommand{\heading}[1]{\vspace{4pt}\noindent\textbf{#1}\enspace}
% \newcommand{\ttt}[1]{\texttt{#1}}
\newcommand{\ttt}[1]{\texttt{#1}}
\newcommand{\ttiny}[1]{\texttt{#1}}
\newcommand{\tti}[1]{\texttt{\scriptsize #1}}
\newcommand{\spol}[1]{\scriptsize{\sc#1}}
\newcommand{\cf}[1]{(\emph{Cf}.\S\ref{#1})}
\newcommand{\sx}[1]{(\S\ref{#1})}
\newcommand{\unix}{{\scshape Unix}\xspace}
\newcommand{\sys}{{\scshape Opaque}\xspace}

% For comments
\newcommand{\eat}[1]{}
\newcommand{\TODO}[1]{\hl{\textbf{TODO:} #1}\xspace}
\newcommand{\todo}[1]{\hl{#1}\xspace}
\newcommand{\fixme}[1]{{\color{red}#1}}
\newcommand{\nv}[1]{[{\color{cyan}nv: #1}]}
\newcommand{\kk}[1]{[{\color{magenta}kk: #1}]}
\newcommand{\review}[1]{{\color{red}#1}}
\newcommand{\tr}[1]{} %% Text and comments for technical report
\newcommand{\str}{{\color{red}\textbf{\ttt{*}}}}

\setlist{noitemsep,leftmargin=10pt,topsep=2pt,parsep=2pt,partopsep=2pt}

\begin{document}

\title{Something Something}

\begin{abstract}
This paper presents \sys, a system for parallelizing \unix shell scripts. 
To address performance and correctness concerns 
\sys introduces an ahead-of-time/just-in-time compilation hybrid to address key runtime challenges specific to the \unix shell---\eg parameter and variable expansion, command substitution, and word splitting.
A set of high-performance primitives such as a windowed \ttt{drain}er and token-based \ttt{split}ter offer further  performance and correctness improvements.
A resource-aware superoptimization component explores the trade-off space to identify configurations 
We evaluate \sys across a variety of hardware and software configurations using an extensive suite of more than 3K scripts including all 418 locale-independent tests from the POSIX suite.
% (1) all 418 locale-independent tests from the POSIX suite,
% (2) 112 scripts from the literature,
% (3) 
\sys's execution produces results that are correct with respect to sequential execution, significantly speeding up parallelizable scripts with sub-second elasticity characteristics. % important for interactive use.
\end{abstract}

\maketitle

\section{Introduction}
\label{intro}

The \unix shell is an environment---often interactive---for composing scripts written in a plethora of programming languages.
This language-agnosticism, coupled with \unix's toolbox philosophy~\cite{mcilroy1978unix}, makes the shell the primary choice for specifying succinct and simple scripts for data processing, system orchestration, and other automation tasks.

\section{Example \& Overview}
\label{overview}

\section{Evaluation}
\label{eval}

\bibliographystyle{ACM-Reference-Format}
{\small
  \bibliography{../EuroSys21/bib}
}

\end{document}
