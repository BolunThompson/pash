%% For double-blind review submission, w/o CCS and ACM Reference (max submission space)
\documentclass[sigplan,10pt,review,anonymous]{acmart}
\settopmatter{printfolios=false,printccs=false,printacmref=false}
%% For double-blind review submission, w/ CCS and ACM Reference
%\documentclass[sigplan,review,anonymous]{acmart}\settopmatter{printfolios=true}
%% For single-blind review submission, w/o CCS and ACM Reference (max submission space)
%\documentclass[sigplan,review]{acmart}\settopmatter{printfolios=true,printccs=false,printacmref=false}
%% For single-blind review submission, w/ CCS and ACM Reference
%\documentclass[sigplan,review]{acmart}\settopmatter{printfolios=true}
%% For final camera-ready submission, w/ required CCS and ACM Reference
%\documentclass[sigplan]{acmart}\settopmatter{}

% \acmSubmissionID{248}

\usepackage{enumitem}
\usepackage{booktabs}
\usepackage{amssymb}
\usepackage{soul}
\usepackage{xspace}
\usepackage{color}
\usepackage{xcolor}
\usepackage{upquote}
\usepackage{listings}
\usepackage{amsmath}
\usepackage{cleveref}
\usepackage{wrapfig}
\usepackage{syntax}

\usepackage{tikz}
\usetikzlibrary{arrows,automata,shapes.misc,shapes.geometric,positioning}

\captionsetup[figure]{font=footnotesize,name={Fig.},labelfont={bf, footnotesize}}
\captionsetup[table]{font=footnotesize,name={Tab.},labelfont={bf, footnotesize}, skip=2pt, aboveskip=2pt}
\captionsetup{font=footnotesize,labelfont={bf, footnotesize}, belowskip=2pt}

\newcommand{\eg}{{\em e.g.}, }
\newcommand{\ie}{{\em i.e.}, }
\newcommand{\etc}{{\em etc.}\xspace}
\newcommand{\vs}{{\em vs.} }
\newcommand{\cmpn}{compartmentalization} 
\newcommand{\heading}[1]{\vspace{4pt}\noindent\textbf{#1}\enspace}
\newcommand{\ttt}[1]{\texttt{\small #1}}
\newcommand{\ttiny}[1]{\texttt{\scriptsize #1}}
\newcommand{\tti}[1]{\texttt{\scriptsize #1}}
\newcommand{\spol}[1]{\scriptsize{\sc#1}}
\newcommand{\pol}[1]{\texttt{\small {\color{purple}#1}}}
\newcommand{\rf}[1]{\ref{#1}}
\newcommand{\wka}{\ttt{a\textsubscript{1}}}
\newcommand{\wkq}{\ttt{q\textsubscript{1-4}}}

\newcommand{\cn}[1]{\mbox{\textcircled{\footnotesize #1}}}
\newcommand{\tcn}[1]{\mbox{\textcircled{\scriptsize #1}}}

\newcommand{\pur}{\cn{\textsc{P}}\xspace}
\newcommand{\sta}{\cn{\textsc{S}}\xspace}
\newcommand{\dfs}{\cn{\textsc{F}}\xspace}
\newcommand{\sid}{\cn{\textsc{E}}\xspace}
\newcommand{\irr}{\cn{\textsc{I}}\xspace}

\newcommand{\tpur}{\tcn{\textsc{P}}\xspace}
\newcommand{\tsta}{\tcn{\textsc{S}}\xspace}
\newcommand{\tdfs}{\tcn{\textsc{F}}\xspace}
\newcommand{\tsid}{\tcn{\textsc{E}}\xspace}
\newcommand{\tirr}{\tcn{\textsc{I}}\xspace}

% For comments
\newcommand{\eat}[1]{}
\newcommand{\TODO}[1]{\hl{\textbf{TODO:} #1}\xspace}
\newcommand{\todo}[1]{\hl{#1}\xspace}
\newcommand{\nv}[1]{[{\color{cyan}#1 --- nv}]}
\newcommand{\kk}[1]{[{\color{magenta}#1 --- kk}]}
\newcommand{\km}[1]{[{\color{blue}#1 --- km}]}
\newcommand{\review}[1]{{\color{red}#1}}
\newcommand{\tr}[1]{} %% Text and comments for technical report

\newcommand{\kstar}{^{\textstyle *}}
\newcommand{\eps}{\varepsilon}

\definecolor{editorGray}{rgb}{0.95, 0.95, 0.95}
\definecolor{editorOcher}{rgb}{1, 0.5, 0} % #FF7F00 -> rgb(239, 169, 0)
\definecolor{editorGreen}{rgb}{0, 0.5, 0} % #007C00 -> rgb(0, 124, 0)

\definecolor{cdb}{rgb}{0.37, 0.62, 0.63} % cadet blue

\lstdefinelanguage{sh}{
  morekeywords={for, in, do, done, \|},
  keywordstyle=\color{purple}\ttfamily,
  % ndkeywords={curl, grep, wget, awk, xargs, find, nc, mdc, gunzip, cut, sort, head, join},
  ndkeywordstyle=\color{black}\ttfamily\bfseries,
  identifierstyle=\color{black},
  sensitive=false,
  comment=[l]{\#},
  commentstyle=\color{lightgray},
% morecomment=[s]{/\\*\\*, \\*/},
  stringstyle=\color{darkgray}\ttfamily,
  morestring=[b]',
  morestring=[b]",
% numbersep=1pt,
% numberstyle=\footnotesize\bf\color{gray},   % the style that is used for the line-numbers
  abovecaptionskip=0pt,
  aboveskip=0pt,
  belowcaptionskip=0pt,
  belowskip=0pt,
  frame=none                     % adds a frame around the code
% moredelim=[s][\color{gray}]{c:}{>},
% moredelim=[s][\color{orange}]{/*}{/}
}

\lstset{ %
  backgroundcolor=\color{white},   % choose the background color; you must add \usepackage{color} or \usepackage{xcolor}
  basicstyle=\small\ttfamily,  % the size of the fonts that are used for the code
  upquote=true,
  captionpos=b,                    % sets the caption-position to bottom
% frame=B,                    % adds a frame around the code
  numbers=left,                    % where to put the line-numbers; possible values are (none, left, right)
  numbersep=2pt,                   % how far the line-numbers are from the code
  numberstyle=\tiny\color{gray},   % the style that is used for the line-numbers
  rulecolor=\color{black},         % if not set, the frame-color may be changed on line-breaks within not-black text (e.g. comments (green here))
  framerule=0pt,
	xleftmargin=0pt,
	xrightmargin=0pt,
	breakindent=0pt,
  aboveskip=0pt,
  framesep=0pt,
  abovecaptionskip=0pt,
  aboveskip=0pt,
  belowcaptionskip=0pt,
  belowskip=0pt,
  frame=none,
  framexbottommargin=0pt,
  resetmargins=true
}



%% Conference information
%% Supplied to authors by publisher for camera-ready submission;
%% use defaults for review submission.
\acmConference[PL'18]{ACM SIGPLAN Conference on Programming Languages}{January 01--03, 2018}{New York, NY, USA}
\acmYear{2018}
\acmISBN{} % \acmISBN{978-x-xxxx-xxxx-x/YY/MM}
\acmDOI{} % \acmDOI{10.1145/nnnnnnn.nnnnnnn}
\startPage{1}

%% Copyright information
%% Supplied to authors (based on authors' rights management selection;
%% see authors.acm.org) by publisher for camera-ready submission;
%% use 'none' for review submission.
\setcopyright{none}
%\setcopyright{acmcopyright}
%\setcopyright{acmlicensed}
%\setcopyright{rightsretained}
%\copyrightyear{2018}           %% If different from \acmYear

%% Bibliography style
\bibliographystyle{ACM-Reference-Format}
%% Citation style
%\citestyle{acmauthoryear}  %% For author/year citations
%\citestyle{acmnumeric}     %% For numeric citations
%\setcitestyle{nosort}      %% With 'acmnumeric', to disable automatic
                            %% sorting of references within a single citation;
                            %% e.g., \cite{Smith99,Carpenter05,Baker12}
                            %% rendered as [14,5,2] rather than [2,5,14].
%\setcitesyle{nocompress}   %% With 'acmnumeric', to disable automatic
                            %% compression of sequential references within a
                            %% single citation;
                            %% e.g., \cite{Baker12,Baker14,Baker16}
                            %% rendered as [2,3,4] rather than [2-4].


%%%%%%%%%%%%%%%%%%%%%%%%%%%%%%%%%%%%%%%%%%%%%%%%%%%%%%%%%%%%%%%%%%%%%%
%% Note: Authors migrating a paper from traditional SIGPLAN
%% proceedings format to PACMPL format must update the
%% '\documentclass' and topmatter commands above; see
%% 'acmart-pacmpl-template.tex'.
%%%%%%%%%%%%%%%%%%%%%%%%%%%%%%%%%%%%%%%%%%%%%%%%%%%%%%%%%%%%%%%%%%%%%%


%% Some recommended packages.
\usepackage{booktabs}   %% For formal tables:
                        %% http://ctan.org/pkg/booktabs
\usepackage{subcaption} %% For complex figures with subfigures/subcaptions
                        %% http://ctan.org/pkg/subcaption


\begin{document}

%% Title information
\title{Dish: Distribution-oblivious Shell Scripting}         %% [Short Title] is optional;
% \titlenote{with title note}             %% \titlenote is optional;
%                                         %% can be repeated if necessary;
%                                         %% contents suppressed with 'anonymous'
% \subtitle{Subtitle}                     %% \subtitle is optional
% \subtitlenote{with subtitle note}       %% \subtitlenote is optional;
                                        %% can be repeated if necessary;
                                        %% contents suppressed with 'anonymous'


%% Author information
%% Contents and number of authors suppressed with 'anonymous'.
%% Each author should be introduced by \author, followed by
%% \authornote (optional), \orcid (optional), \affiliation, and
%% \email.
%% An author may have multiple affiliations and/or emails; repeat the
%% appropriate command.
%% Many elements are not rendered, but should be provided for metadata
%% extraction tools.

%% Author with single affiliation.
\author{First1 Last1}
\authornote{with author1 note}          %% \authornote is optional;
                                        %% can be repeated if necessary
\orcid{nnnn-nnnn-nnnn-nnnn}             %% \orcid is optional
\affiliation{
  \position{Position1}
  \department{Department1}              %% \department is recommended
  \institution{Institution1}            %% \institution is required
  \streetaddress{Street1 Address1}
  \city{City1}
  \state{State1}
  \postcode{Post-Code1}
  \country{Country1}                    %% \country is recommended
}
\email{first1.last1@inst1.edu}          %% \email is recommended

%% Author with two affiliations and emails.
\author{First2 Last2}
\authornote{with author2 note}          %% \authornote is optional;
                                        %% can be repeated if necessary
\orcid{nnnn-nnnn-nnnn-nnnn}             %% \orcid is optional
\affiliation{
  \position{Position2a}
  \department{Department2a}             %% \department is recommended
  \institution{Institution2a}           %% \institution is required
  \streetaddress{Street2a Address2a}
  \city{City2a}
  \state{State2a}
  \postcode{Post-Code2a}
  \country{Country2a}                   %% \country is recommended
}
\email{first2.last2@inst2a.com}         %% \email is recommended
\affiliation{
  \position{Position2b}
  \department{Department2b}             %% \department is recommended
  \institution{Institution2b}           %% \institution is required
  \streetaddress{Street3b Address2b}
  \city{City2b}
  \state{State2b}
  \postcode{Post-Code2b}
  \country{Country2b}                   %% \country is recommended
}
\email{first2.last2@inst2b.org}         %% \email is recommended

\newcommand{\cf}[1]{(\emph{Cf}.\S\ref{#1})}
\newcommand{\sx}[1]{(\S\ref{#1})}
\newcommand{\sys}{{\scshape Dish}\xspace}
\newcommand{\unix}{{\scshape Unix}\xspace}

\setlist{noitemsep,leftmargin=10pt,topsep=2pt,parsep=2pt,partopsep=2pt}

\begin{abstract}
  one two three
\end{abstract}

%% 2012 ACM Computing Classification System (CSS) concepts
%% Generate at 'http://dl.acm.org/ccs/ccs.cfm'.
\begin{CCSXML}
<ccs2012>
<concept>
<concept_id>10011007.10011006.10011008</concept_id>
<concept_desc>Software and its engineering~General programming languages</concept_desc>
<concept_significance>500</concept_significance>
</concept>
<concept>
<concept_id>10003456.10003457.10003521.10003525</concept_id>
<concept_desc>Social and professional topics~History of programming languages</concept_desc>
<concept_significance>300</concept_significance>
</concept>
</ccs2012>
\end{CCSXML}

\ccsdesc[500]{Software and its engineering~General programming languages}
\ccsdesc[300]{Social and professional topics~History of programming languages}
%% End of generated code

% \keywords{keyword1, keyword2, keyword3}  %% \keywords are mandatory in final camera-ready submission

\maketitle

\section{Introduction}
\label{intro}

Distributed systems offer significant benefits over their centralized counterparts---for example, they can speed up expensive computations or can process data that would not fit into any single machine.
Despite these benefits, their development remains different from and significantly more difficult than their centralized counterparts.
Whereas anyone can quickly stitch together a Bash script to compute on a single computer, 
  % domain-experts routinely glue scripts together to process and share data. % without the help of a computing expert.
   scaling out to multiple ones requires expert labor around ``point'' solutions with expensive setups, restricted programming interfaces, and exorbitant composition costs~\cite{taurus:14, dios:13, andromeda:15, pywren:17, futuredata:18, nefele:18}.
% 
To understand this sharp contrast, consider a Bash pipeline calculating term frequencies over a set of inputs:


Contributions:

* 
Extends beyond linear pipelines to full programs 
that comply with the POSIX shell specification; while \sys does not optimize
everything---it certainly cannot---, it does 

* Planner activations, the ability to call 

* A standard library of semantics-aware wrappers for 

* Augment data flows to allow performance optimizations and fault-tolerance

\section{Background}
\label{bg}

To understand the limitations of pipelines and the opportunities for
parallelization, consider the following pipeline

\begin{lstlisting}[language=sh, float=h, numbers=none, escapeinside={($}{$)}]
find . -name '*.sh' | xargs cat | wc -l > out.t
\end{lstlisting}

Unfortunately, this pipeline meshes the results from `cat` leading to a single
count from all files. What the developer actually intended was the following:

\begin{lstlisting}[language=sh, float=h, numbers=none, escapeinside={($}{$)}]
$(
for f in $(find . -name '*.sh'); do
 cat ${f} | wc -l
done
) > out.t
\end{lstlisting}

This remains trivial to parallelize, as none of the loop-internal steps depends
on the input, so loop iterations can run in parallel.

Similar for
  * enclaves 
  * sieve
  * mapping max temperatures to years, etc.

Different from prior systems, \sys's goal is to ensure correctness first and
then improve performance.

\section{Evaluation}

\subsection{POSIX test-suite}

\subsection{Pipelines and scripts}

\subsection{Other pipelines}

Rewrite pipelines expressed in other languages / scripts

\paragraph{Conclusion}
\begin{acks}
  % Dumping people so that we don't forget
  % 
  This material is based upon work supported by the
  \grantsponsor{GS100000001}{National Science
    Foundation}{http://dx.doi.org/10.13039/100000001} under Grant
  No.~\grantnum{GS100000001}{nnnnnnn} and Grant
  No.~\grantnum{GS100000001}{mmmmmmm}.  Any opinions, findings, and
  conclusions or recommendations expressed in this material are those
  of the author and do not necessarily reflect the views of the
  National Science Foundation.
\end{acks}


%% Bibliography
\bibliography{./bib}


%% %% Appendix
%% \appendix
%% \section{Scripts used in the evaluation}

%% This appendix contains the source code of the scripts used in the evaluation of
%% the \sys. They are part of the codebase (released as open source with the camera
%% ready), and are provided here only to aid the reviewers.


\end{document}
